\documentclass[a4paper,12pt] {book} 
     
\usepackage{booktabs,longtable,rotating} % long tables are split to other lines %for rotating tables
\usepackage{expl3}
\usepackage{xparse}
\usepackage{xstring}
\usepackage{relsize} % sizes relative to current
\usepackage{array,warpcol} % Alignment on decimal point entries in tables
\usepackage{multirow} % muli rows (merge) in tables
\usepackage{fixme} %
\usepackage{etex}
\reserveinserts{28}
\usepackage{etoolbox}

%\newcommand{\widthoftable}{\dimexpr\textwidth-2\tabcolsep\relax}
%\def \BASECASSFirstColumnWidth {2cm}
%\def \BASECASSSecondColumnWidth {2cm}
%\def \BASECASSThirdColumnWidth {8cm}
%\def \BASECASSFirstAndSecondColumnWidth {4cm}
%\def \BASECASSSecondAndThirdColumnWidth {11cm}

\begin{document} 





\newlength{\columnWidthForTwoColumnsOne}
\newlength{\columnWidthForTwoColumnsTwo}
\newlength{\columnWidthForTwoColumnsThree}
%\newlength{\columnWidthForTwoColumnsOneTwo}
\newlength{\columnWidthForTwoColumnsTwoThree}
\newlength{\columnWidthForTwoColumnsOneTwoThree}

\setlength{\columnWidthForTwoColumnsOne}{2.5cm-\tabcolsep-\tabcolsep}
\setlength{\columnWidthForTwoColumnsTwo}{2.5cm-\tabcolsep-\tabcolsep}
\setlength{\columnWidthForTwoColumnsThree}{8.5cm}
%\setlength{\columnWidthForTwoColumnsOneTwo}{4cm-\tabcolsep-\tabcolsep-\tabcolsep-\tabcolsep}
\setlength{\columnWidthForTwoColumnsTwoThree}{11cm-\tabcolsep-\tabcolsep}
\setlength{\columnWidthForTwoColumnsOneTwoThree}{13.5cm-\tabcolsep-\tabcolsep-\tabcolsep-\tabcolsep}

\newlength{\columnWidthForThreeColumnsOne}
\newlength{\columnWidthForThreeColumnsTwo}
\newlength{\columnWidthForThreeColumnsThree}
\newlength{\columnWidthForThreeColumnsOneTwo}
\newlength{\columnWidthForThreeColumnsTwoThree}
\newlength{\columnWidthForThreeColumnsOneTwoThree}

\setlength{\columnWidthForThreeColumnsOne}{2.5cm-\tabcolsep-\tabcolsep}
\setlength{\columnWidthForThreeColumnsTwo}{2.5cm-\tabcolsep-\tabcolsep}
\setlength{\columnWidthForThreeColumnsThree}{8.5cm-\tabcolsep-\tabcolsep}
\setlength{\columnWidthForThreeColumnsOneTwo}{5cm-\tabcolsep-\tabcolsep-\tabcolsep-\tabcolsep}
\setlength{\columnWidthForThreeColumnsTwoThree}{11cm-\tabcolsep-\tabcolsep-\tabcolsep-\tabcolsep}
\setlength{\columnWidthForThreeColumnsOneTwoThree}{13.5cm-\tabcolsep-\tabcolsep-\tabcolsep-\tabcolsep-\tabcolsep-\tabcolsep}

% to set row height, from https://tex.stackexchange.com/questions/84524/how-to-make-a-row-in-a-table-shorter

\makeatletter
\newsavebox\saved@arstrutbox
\newcommand*{\setarstrut}[1]{%
  \noalign{%
    \begingroup
      \global\setbox\saved@arstrutbox\copy\@arstrutbox
      #1%
      \global\setbox\@arstrutbox\hbox{%
        \vrule \@height\arraystretch\ht\strutbox
               \@depth\arraystretch \dp\strutbox
               \@width\z@
      }%
    \endgroup
  }%
}
\newcommand*{\restorearstrut}{%
  \noalign{%
    \global\setbox\@arstrutbox\copy\saved@arstrutbox
  }%
}
\makeatother

% anyadimos esto a la primera columna >{\hspace{0pt}}   para q lo anyada a todas las primeras columnas
% esto hace q TeX no considere a la primera palabra cOmo la primera palabra
% es importante pq TeX no hace hypenathion de la primera palabra
% info aqi'i: https://tex.stackexchange.com/questions/198325/wrap-word-in-table-cell

\newcommand{\basecassTitle}[3]{%
%\begin{tabular}{ @{\extracolsep{\fill}} | p{\columnWidthForTwoColumnsOne}  |  p{\columnWidthForTwoColumnsTwo} |  p{\columnWidthForTwoColumnsThree} | @{}}
%\begin{tabular}{ @{\extracolsep{\fill}} | p{\columnWidthForTwoColumnsOne}  |  p{\columnWidthForTwoColumnsTwo} |  p{\columnWidthForTwoColumnsThree} | @{}}
%    \hline
%%      &  &   \\[-2ex] % hace falta anyadir una linea pq respeta el espacio de cols con ella
%      \multicolumn{3}{| p{\columnWidthForTwoColumnsOneTwoThree} |}{ }\\
%      \multicolumn{3}{| c |}{ #1 }\\
%      \multicolumn{3}{| p{\columnWidthForTwoColumnsOneTwoThree} |}{  }\\
%    \hline
%%      Name: & \multicolumn{2}{ c |}{ {\relsize{-1} #2 } } \\
%%      Description: & \multicolumn{2}{ c |}{ {\relsize{-1} #3 } } \\
%      Name: & \multicolumn{2}{ p{\columnWidthForThreeColumnsTwoThree} |}{ #2 } \\
%      Description: & \multicolumn{2}{ p{\columnWidthForThreeColumnsTwoThree} |}{ #3 } \\
%    \hline
%\end{tabular}%
\begin{tabular}{ @{\extracolsep{\fill}} | >{\hspace{0pt}} p{\columnWidthForTwoColumnsOne}  |  >{\hspace{0pt}} p{\columnWidthForTwoColumnsTwo} |  p{\columnWidthForTwoColumnsThree} | @{}}
    \hline
      \multicolumn{3}{| p{\columnWidthForTwoColumnsOneTwoThree} |}{ }\\
      \multicolumn{3}{| c |}{ #1 }\\
      \multicolumn{3}{| p{\columnWidthForTwoColumnsOneTwoThree} |}{ }\\
    \hline
      Name: & \multicolumn{2}{ p{\columnWidthForTwoColumnsTwoThree} |}{ #2 } \\
      Description: & \multicolumn{2}{ p{\columnWidthForTwoColumnsTwoThree} |}{ #3 } \\
    \hline
\end{tabular}%
 }

\newcommand{\ChallengeDomain}[5]{%
\begin{tabular}{ @{\extracolsep{\fill}} | >{\hspace{0pt}} p{\columnWidthForThreeColumnsOne}  |  >{\hspace{0pt}} p{\columnWidthForThreeColumnsTwo} |  p{\columnWidthForThreeColumnsThree} | @{}}
    \hline
      \multicolumn{3}{| c |}{\multirow{2}{*} {Challenge space } } \\
      \multicolumn{3}{| c |}{ } \\
    \hline
      \parbox[t]{2mm}{\multirow{2}{*}{\parbox{\columnWidthForThreeColumnsOne}{Base \\ problem:}}} & Type: & #1 \\
      \cline{2-3}
      & Size: & #2  \\
      \hline
      
      \parbox[t]{2mm}{\multirow{2}{*}{\rotatebox[origin=c]{90}{\parbox{\columnWidthForThreeColumnsOne}{CAPTCHA \\ problem:}}}} & Domain: & #3 \\
      \cline{2-3}
      & Size: & #4 \\
      \cline{2-3}
      & Distribution: & #5  \\

   \hline
\end{tabular}%
 }

\newcommand{\basecassAnswerDomain}[5]{%
%\begin{tabular}{ @{\extracolsep{\fill}} | p{\columnWidthForThreeColumnsOne}  |  p{\columnWidthForThreeColumnsTwo} |  p{\columnWidthForThreeColumnsThree} | @{}}
\begin{tabular}{ @{\extracolsep{\fill}} | >{\hspace{0pt}} p{\columnWidthForThreeColumnsOne}  |  >{\hspace{0pt}} p{\columnWidthForThreeColumnsTwo} |  p{\columnWidthForThreeColumnsThree} | @{}}
   \hline
%      \rowcolor{White}
      \multicolumn{3}{| c |}{\multirow{2}{*}{Answer space } } \\
%      \multicolumn{3}{| c |}{ } \\
      \multicolumn{3}{| p{\columnWidthForTwoColumnsOneTwoThree} |}{ } \\

%    \rowcolor{LightYellow}
    
    \hline
      Maximum Range: & \multicolumn{2}{ p{\columnWidthForThreeColumnsTwoThree} |}{ #1 } \\
      \hline
      Range: & \multicolumn{2}{ p{\columnWidthForThreeColumnsTwoThree} |}{ #2 } \\
      \hline
      Ratio: & {#3} & {#4} \\
      \hline
      Distribution: & \multicolumn{2}{ p{\columnWidthForThreeColumnsTwoThree} |}{ #5 } \\
   \hline
\end{tabular}%
 }

\newcommand{\basecassChallengeAnswerDomainConclusions}[4]{%
%\begin{tabular}{ @{\extracolsep{\fill}} | p{\columnWidthForThreeColumnsOne}  |  p{\columnWidthForThreeColumnsTwo} |  p{\columnWidthForThreeColumnsThree} | @{}}
\begin{tabular}{ @{\extracolsep{\fill}} | >{\hspace{0pt}} p{\columnWidthForThreeColumnsOne}  |  >{\hspace{0pt}} p{\columnWidthForThreeColumnsTwo} |  p{\columnWidthForThreeColumnsThree} | @{}}
   \hline
%      \rowcolor{White}
%      \rowcolor{White}
      \multicolumn{3}{| c |}{\multirow{2}{*}{Challenge space \& answer space conclusions } } \\
%      \multicolumn{3}{| c |}{ } \\
      \multicolumn{3}{| p{\columnWidthForTwoColumnsOneTwoThree} |}{ } \\


%    \rowcolor{LightGreen}
    
    \hline
      Is attack possible: & #1 & #2  \\
      \hline
      Description: & \multicolumn{2}{ p{\columnWidthForThreeColumnsTwoThree} |}{ #3 } \\
      \hline
      Success: & \multicolumn{2}{ p{\columnWidthForThreeColumnsTwoThree} |}{ #4 } \\
    \hline
\end{tabular}%
 }

\ExplSyntaxOn
\newcommand{\basecassSubtableIntForTwoColumns}[1]{ % elems
	\clist_map_inline:nn { #1 } { & \multicolumn{2}{ p{\columnWidthForTwoColumnsTwoThree} |}{ ##1} \\  }
}
\ExplSyntaxOff

\ExplSyntaxOn
\newcommand{\basecassSubtableForTwoColumns}[3]{ % {tItulo} {elem1} {elems2,elems3,...}
    \hline
    %\setbox0\hbox{\tabular{@{}l} \hspace{0pt} #1 \endtabular}
    % note that length of multirow should be \clist_count:n {#3} + 1
    %\parbox[t]{2mm}{\multirow{\clist_count:n {#3}}{*}{\rotatebox[origin=c]{90}{\parbox{\columnWidthForTwoColumnsOne}{#1}}}} & \multicolumn{2}{ p{\columnWidthForTwoColumnsTwoThree} |}{ #2 } \\
    \parbox[t]{2mm}{\multirow{\clist_count:n {#3}}{*}
    {
     %\rule{0pt}{\dimexpr(\wd0-\normalbaselineskip+\tabcolsep+\tabcolsep)}
     %\rotatebox{90}{\rlap{\usebox0}}
     \rotatebox{90}{ \tabular{@{}l} \hspace{0pt} #1 \endtabular }}
    } & \multicolumn{2}{ p{\columnWidthForTwoColumnsTwoThree} |}{ #2 } \\
    \basecassSubtableIntForTwoColumns{#3}
    & \multicolumn{2}{ p{\columnWidthForTwoColumnsTwoThree} |}{   } \\
}
\ExplSyntaxOff




\newcommand{\basecassMetrics}[8]{%
%\begin{tabular}{ @{\extracolsep{\fill}} | p{\columnWidthForTwoColumnsOne}  |  p{\columnWidthForTwoColumnsTwo} |  p{\columnWidthForTwoColumnsThree} | @{}}
\begin{tabular}{ @{\extracolsep{\fill}} | >{\hspace{0pt}} p{\columnWidthForTwoColumnsOne}  |  >{\hspace{0pt}} p{\columnWidthForTwoColumnsTwo} |  p{\columnWidthForTwoColumnsThree} | @{}}
   \hline
%      \rowcolor{White}
      \multicolumn{3}{| c |}{\multirow{2}{*}{Metrics } } \\
%      \multicolumn{3}{| c |}{ } \\
      \multicolumn{3}{| p{\columnWidthForTwoColumnsOneTwoThree} |}{ } \\

    \hline
      Denoising: & \multicolumn{2}{ p{\columnWidthForTwoColumnsTwoThree} |}{ #1 } \\

    \hline
      Pre-processing: & \multicolumn{2}{ p{\columnWidthForTwoColumnsTwoThree} |}{ #2 } \\

    \basecassSubtableForTwoColumns{Generic}{#3}{#4}
    \basecassSubtableForTwoColumns{Order}{#5}{#6}
    \basecassSubtableForTwoColumns{Specific/ \\ Tailored}{#7}{#8}

	\hline
\end{tabular}%
 }




% estas macros ...Two son mejores q las sencillas: reciben cOmo Ultimo parAmetro la especificaciOn de la columna, lo que evita quela columna cambie de ancho si el contenido es muy largo
% si este error ocurre, cambiar las llamadas a las macros originales pa q sean a estas ...Two

\ExplSyntaxOn
\newcommand{\basecassSubtableIntTwoForTwoColumns}[2]{ % elems
	\clist_map_inline:nn { #1 } { & \multicolumn{2}{ #2 }{ ##1} \\  }
}
\ExplSyntaxOff

\ExplSyntaxOn
\newcommand{\basecassSubtableTwoForTwoColumns}[4]{ % {tItulo} {elem1} {elems2,elems3,...} p{\columnWidthForTwoColumnsThree}
    %\setbox0\hbox{\tabular{@{}l} \hspace{0pt} #1 \endtabular}
    \hline
    % note that length of multirow should be \clist_count:n {#3} + 1
    \parbox[t]{2mm}{\multirow{\clist_count:n {#3}}{*}
    {
     %\rule{0pt}{\dimexpr(\wd0-\normalbaselineskip+\tabcolsep+\tabcolsep)}
     %\rotatebox{90}{\rlap{\usebox0}}
     \rotatebox[origin=c]{90}{\parbox{\columnWidthForTwoColumnsOne}{ \tabular{@{}l} \hspace{0pt} #1 \endtabular }
    }}} & \multicolumn{2}{ #4 }{ #2 } \\
    \basecassSubtableIntTwoForTwoColumns{#3}{#4}
    & \multicolumn{2}{ #4 }{   } \\
}
\ExplSyntaxOff









\makeatletter
\newcommand\basecassTableLineForTwoColumns[1]{%
    \forcsvlist{\basecassTableLineForTwoColumns@item}{#1}
}
\newcommand\basecassTableLineForTwoColumns@item[1]{ \multicolumn{3}{| p{\columnWidthForTwoColumnsOneTwoThree} |}{#1} \\ }
\makeatother


\newcommand{\basecassTestOfMetrics}[5]{%
%\begin{tabular}{ @{\extracolsep{\fill}} | p{\columnWidthForTwoColumnsOne}  |  p{\columnWidthForTwoColumnsTwo} |  p{\columnWidthForTwoColumnsThree} | @{}}
\begin{tabular}{ @{\extracolsep{\fill}} | >{\hspace{0pt}} p{\columnWidthForTwoColumnsOne}  |  >{\hspace{0pt}} p{\columnWidthForTwoColumnsTwo} |  p{\columnWidthForTwoColumnsThree} | @{}}
   \hline
%      \rowcolor{White}
      \multicolumn{3}{| c |}{\multirow{2}{*}{ Test of metrics } } \\
%      \multicolumn{3}{| c |}{ } \\
      \multicolumn{3}{| p{\columnWidthForTwoColumnsOneTwoThree} |}{ } \\
      
    \hline      
	\basecassTableLineForTwoColumns{ #1 }
    \hline                                                                                                                                                                                                                                                                                                                                                  
	\basecassTableLineForTwoColumns{ #2 }

    \hline                                                                                                                                                                                                                                                                                                                                                  
      Is attack possible: & \multicolumn{2}{ p{\columnWidthForTwoColumnsTwoThree} |}{ #3 } \\
      \hline
      Description: & \multicolumn{2}{ p{\columnWidthForTwoColumnsTwoThree} |}{ #4 } \\
      \hline
      Success: & \multicolumn{2}{ p{\columnWidthForTwoColumnsTwoThree} |}{ #5 } \\

	\hline
\end{tabular}%
 }

%\newcommand{\basecassTestOfMetrics}[5][]{%
%
%\begin{tabular}{ @{\extracolsep{\fill}} | p{\columnWidthForTwoColumnsOne}  |  p{\columnWidthForTwoColumnsTwo} |  p{\columnWidthForTwoColumnsThree} | @{}}
%   \hline
%      \rowcolor{White}
%      \multicolumn{3}{| c |}{\multirow{2}{*}{ Test of metrics } } \\
%      \multicolumn{3}{| p{\columnWidthForTwoColumnsOneTwoThree} |}{ } \\
%
%    \hline      
%	\basecassTableLine{ #1 }    
%    \hline                                                                                                                                                                                                                                                                                                                                                  
%	\basecassTableLine{ #2 }    
%
%    \hline                                                                                                                                                                                                                                                                                                                                                  
%      Is attack possible: & \multicolumn{2}{ p{\columnWidthForTwoColumnsTwoThree} |}{ #3 } \\
%      \hline
%      Description: & \multicolumn{2}{ p{\columnWidthForTwoColumnsTwoThree} |}{ #4 } \\
%      \hline
%      Success: & \multicolumn{2}{ p{\columnWidthForTwoColumnsTwoThree} |}{ #5 } \\
%
%    \hline                                                                                                                                                                                                                                                                                                                                                  
%\end{tabular}%
% }

\newcommand{\basecassDataPreparation}[3]{% 

\begin{tabular}{ @{\extracolsep{\fill}} | >{\hspace{0pt}} p{\columnWidthForThreeColumnsOne}  |  >{\hspace{0pt}} p{\columnWidthForThreeColumnsTwo} |  p{\columnWidthForThreeColumnsThree} | @{}}
   \hline
%      \rowcolor{White}
      \multicolumn{3}{| c |}{\multirow{2}{*}{ Data preparation } } \\
      \multicolumn{3}{| c |}{ } \\

      \hline
      \parbox[t]{2mm}{\multirow{3}{*}{\rotatebox[origin=c]{90}{\parbox{\columnWidthForThreeColumnsOne}{Training \\ set}}}} & Size: & #1  \\
      \cline{2-3}
      & Balance: & #2  \\ 
      \cline{2-3}
      & Notes: & #3  \\

	\hline
\end{tabular}%
 }



\newcommand{\basecassStatisticalAnalysis}[2]{%

\begin{tabular}{ @{\extracolsep{\fill}} | >{\hspace{0pt}} p{\columnWidthForTwoColumnsOne}  |  >{\hspace{0pt}} p{\columnWidthForTwoColumnsTwo} |  p{\columnWidthForTwoColumnsThree} | @{}}
%\begin{tabular}{ @{\extracolsep{\fill}} | p{\columnWidthForTwoColumnsOne}  |  p{\columnWidthForTwoColumnsTwo} |  p{\columnWidthForTwoColumnsThree} | @{}} 
    \hline                                                                                                                                                                                                                                                                                                                                                  

%      \rowcolor{White}
      \multicolumn{3}{| c |}{\multirow{2}{*}{Statistical analysis } } \\
      \multicolumn{3}{| p{\columnWidthForTwoColumnsOneTwoThree} |}{ } \\
    
    \hline
      Correlations & \multicolumn{2}{ p{\columnWidthForTwoColumnsTwoThree} |}{ #1 } \\
      \hline
      Regressions & \multicolumn{2}{ p{\columnWidthForTwoColumnsTwoThree} |}{ #2 } \\  
 
    \hline
\end{tabular}% 
}

\newcommand{\basecassMLAnalysis}[4]{%

\begin{tabular}{ @{\extracolsep{\fill}} | >{\hspace{0pt}} p{\columnWidthForTwoColumnsOne}  |  >{\hspace{0pt}} p{\columnWidthForTwoColumnsTwo} |  p{\columnWidthForTwoColumnsThree} | @{}} 
    \hline                                                                                                                                                                                                                                                                                                                                                  

%      \rowcolor{White}
      \multicolumn{3}{| c |}{\multirow{2}{*}{ML analysis } } \\
      \multicolumn{3}{| p{\columnWidthForTwoColumnsOneTwoThree} |}{ } \\

%    \rowcolor{LightBlue}

    \hline
      Selection: & \multicolumn{2}{ p{\columnWidthForTwoColumnsTwoThree} |}{ #1 } \\
      \hline
      Best algorithms: & \multicolumn{2}{ p{\columnWidthForTwoColumnsTwoThree} |}{ #2 } \\
      \hline
      Accuracy: & \multicolumn{2}{ p{\columnWidthForTwoColumnsTwoThree} |}{ #3 } \\
      \hline
      $\kappa$-statistic : & \multicolumn{2}{ p{\columnWidthForTwoColumnsTwoThree} |}{ #4 } \\
    \hline
\end{tabular}% 
}

\newcommand{\basecassSMLAttackAndResults}[4]{%

\begin{tabular}{ @{\extracolsep{\fill}} | >{\hspace{0pt}} p{\columnWidthForTwoColumnsOne}  |  >{\hspace{0pt}} p{\columnWidthForTwoColumnsTwo} |  p{\columnWidthForTwoColumnsThree} | @{}} 
      % if phase before leads to an attack
%      \rowcolor{White}
     \hline
      \multicolumn{3}{| c |}{\multirow{2}{*}{S/ML attack \& Results } } \\
      \multicolumn{3}{| p{\columnWidthForTwoColumnsOneTwoThree} |}{ } \\      
      \multicolumn{3}{| c |}{\tiny If previous phase leads to an attack}\\

%    \rowcolor{LightGreen}

    \hline
      Possible?: & \multicolumn{2}{ p{\columnWidthForTwoColumnsTwoThree} |}{ #1 } \\
      \hline
      Description: & \multicolumn{2}{ p{\columnWidthForTwoColumnsTwoThree} |}{ #2 } \\
      \hline
      Success rate: & \multicolumn{2}{ p{\columnWidthForTwoColumnsTwoThree} |}{ #3 } \\
      \hline
      Observations: & \multicolumn{2}{ p{\columnWidthForTwoColumnsTwoThree} |}{ #4 } \\
    \hline
\end{tabular}% 
}

\newcommand{\basecassMLParameterAnalysis}[3]{%

\begin{tabular}{ @{\extracolsep{\fill}} | >{\hspace{0pt}} p{\columnWidthForTwoColumnsOne}  |  >{\hspace{0pt}} p{\columnWidthForTwoColumnsTwo} |  p{\columnWidthForTwoColumnsThree} | @{}} 
    \hline                                                                                                                                                                                                                                                                                                                                                  

      % note: this phase is optional, iff phases before have not lead to a successful attack n
%      \rowcolor{White}
      \multicolumn{3}{| c |}{\multirow{2}{*}{ML vs. parameter analysis } } \\
      \multicolumn{3}{| p{\columnWidthForTwoColumnsOneTwoThree} |}{ } \\      
      \setarstrut{\tiny}%
      \multicolumn{3}{| c |}{\tiny Optional: if and only if phases before not lead to a successful attack}\\
      \multicolumn{3}{| c |}{\tiny  and there is enough data on challenge production parameters}\\
      \restorearstrut


%    \rowcolor{LightOrange}

    \hline

      % for each combination of parameter, value(s), and interesting ML result
      & \multicolumn{2}{ p{\columnWidthForTwoColumnsTwoThree} |}{\tiny For each combination of parameter, value(s), and interesting ML result:} \\
%      \cline{2-3}
      \basecassSubtableForTwoColumns{#1}{#2}{#3}  
%     quedaba mejor el original, al usar bien las tres columnas:
%      \parbox[t]{2mm}{\multirow{4}{*}{\rotatebox[origin=c]{90}{\parbox{\columnWidthForTwoColumnsOne}{\textless Parameter \\ name \textgreater }}}} & value/s: & \textless Description of set of values for the parameter that lead to an interesting result \textgreater \\
%      \cline{2-3}
%      & Best algorithm: & #4  \\
%      \cline{2-3}
%      & Accuracy: &  #5 \\
%      \cline{2-3}
%      & $\kappa$-statistic : & #6  \\
%      \cline{2-3}
%      &  \ldots & \\
      
    \hline
\end{tabular}% 
}

\newcommand{\basecassAttackAndResults}[4]{%

\begin{tabular}{ @{\extracolsep{\fill}} | >{\hspace{0pt}} p{\columnWidthForTwoColumnsOne}  |  >{\hspace{0pt}} p{\columnWidthForTwoColumnsTwo} |  p{\columnWidthForTwoColumnsThree} | @{}} 
    \hline                                                                                                                                                                                                                                                                                                                                                  

      % if phase before leads to an attack
%      \rowcolor{White}
      \multicolumn{3}{| c |}{\multirow{2}{*}{Attack \& Results } } \\
      \multicolumn{3}{| p{\columnWidthForTwoColumnsOneTwoThree} |}{ } \\
      \multicolumn{3}{| c |}{\tiny If previous phase leads to an attack}\\

%    \rowcolor{LightRed}

    \hline
      Possible?: & \multicolumn{2}{ p{\columnWidthForTwoColumnsTwoThree} |}{ #1 } \\
      \hline
      Description: & \multicolumn{2}{ p{\columnWidthForTwoColumnsTwoThree} |}{ #2 } \\
      \hline
      Success rate: & \multicolumn{2}{ p{\columnWidthForTwoColumnsTwoThree} |}{ #3 } \\
      \hline
      Observations: & \multicolumn{2}{ p{\columnWidthForTwoColumnsTwoThree} |}{ #4 } \\
    \hline
\end{tabular}% 
}

\newcommand{\basecassConclusion}[3]{%

\begin{tabular}{ @{\extracolsep{\fill}} | >{\hspace{0pt}} p{\columnWidthForTwoColumnsOne}  |  >{\hspace{0pt}} p{\columnWidthForTwoColumnsTwo} |  p{\columnWidthForTwoColumnsThree} | @{}} 
    \hline
%      \rowcolor{White}
      \multicolumn{3}{| c |}{\multirow{2}{*}{Conclusion } } \\
      \multicolumn{3}{| p{\columnWidthForTwoColumnsOneTwoThree} |}{ } \\

%    \rowcolor{LightGreen}

    \hline
      Weaknesses: & \multicolumn{2}{ p{\columnWidthForTwoColumnsTwoThree} |}{ #1 } \\
      \hline
      Broken?: & \multicolumn{2}{ p{\columnWidthForTwoColumnsTwoThree} |}{ #2 } \\
      \hline
      Work-arounds: & \multicolumn{2}{ p{\columnWidthForTwoColumnsTwoThree} |}{ #3 } \\
    \hline 
\end{tabular}% 
}


%  \begin{longtable}{@{\extracolsep{\fill}}l@{}} %{@{}l@{}}
%
%    
%    \basecassTitleAndChallengeDomain{BASECASS \textless CAPTCHA \textgreater Analysis}{\textless Captcha name and challenge subtype, if many \textgreater}{\textless more detailed description \textgreater}{\textless Basic category of the problem presented\textgreater}{\textless Estimation of base problem size\textgreater}{\textless Detailed description of the specific problem presented by the CAPTCHA\textgreater}{\textless Estimation of size, compared to the base problem, and/or based on possible parameters that influence on the creation of each challenge\textgreater}{\textless Distribution in which each challenge parameter value appears, whether it is uniform or not, and additional data. A Pearson's $\rchi^2$ test might be applied if enough data is available\textgreater} \\
%
%	\basecassAnswerDomain{\textless Theoretical size of the set of possible values to answers\textgreater}{\textless Real size of set of possible answers\textgreater}{\textless Ratio \textgreater}{\textless Ratio (if finite)\textgreater}{\textless Distribution in which each answer value appears. A Pearson's $\rchi^2$ test might be applied if enough data is available\textgreater} \\
%
%	\basecassChallengeAnswerDomainConclusions{Yes/No}{\textless Whether an attack might be possible or not based on the previous findings\textgreater}{\textless How the attack works\textgreater}{\textless Real success rate with which the attack bypasses the CAPTCHA\textgreater} \\
%
%	\basecassMetrics{\textless Whether any denoising technique is used. If so, comment which\textgreater}{\textless Whether any pre-processing technique is used. If so, describe it\textgreater}{\textless General purpose metric used \# 1 \textgreater}{\textless General purpose metric used \# 2 \textgreater, \ldots}{\textless Order metric used \# 1 \textgreater}{\textless Order metric used \# 2 \textgreater, \ldots}{\textless Special metric used \# 1 \textgreater}{\textless Special metric used \# 2 \textgreater, \ldots} \\
%
%%\basecassTestOfMetrics[lineAA=adkakdlsd,textAA=weewrwerw]{uno} \\ 
%
%\basecassTestOfMetrics{ $metric_1$ : \textless Check metric applies to challenges and direct information gain \textgreater, $metric_2$ : \textless Check metric applies to challenges and direct information gain \textgreater , \ldots } { $metric_i$ vs. $metric_j$ : p{\columnWidthForTwoColumnsTwoThree} |}{\textless Check if both present highly correlated results \textgreater , \ldots , \ldots } { \textless Whether an attack might be possible using one of the tested metrics \textgreater } { \textless Which metric the attack uses and how \textgreater } { \textless Real success rate with which the attack bypasses the CAPTCHA\textgreater } \\ 
%	
%\basecassDataPreparation{\textless Number of training examples\textgreater}{\textless How many of them are correct vs. wrong\textgreater}{\textless Optional notes about data cleaning, transformations, data distribution, etc. \textgreater} \\
%
%\basecassStatisticalAnalysis{\textless Most correlated variables with answers and R-factors\textgreater}{\textless Variables that are used in best linear regression, and error\textgreater} \\
%
%\basecassMLAnalysis{\textless Selection criteria for fitness of ML algorithm\textgreater}{\textless List of $N$ best performing ML algorithms\textgreater}{\textless Accuracy of the $N$ best algorithms\textgreater}{\textless $\kappa$-statistic of the $N$ best algorithms\textgreater } \\
%
%\basecassSMLAttackAndResults{\textless Whether an attack based on the previous findings seems possible or not\textgreater}{\textless How the attack works\textgreater}{\textless If so, with which success rate it bypasses the CAPTCHA\textgreater}{\textless Any additional observations\textgreater} \\
%
%% ojo: el primer parAmetro deberIa ser "Parameter \\ name" pero no le gustan los "\\"
%\basecassMLParameterAnalysis{ \textless Parameter name \textgreater }{Value/s: \textless Description of set of values for the parameter that lead to an interesting result \textgreater}{Best algorithm: \textless Best performing ML algorithm\textgreater, Accuracy: \textless Accuracy of the best ML algorithm\textgreater, $\kappa$-statistic : \textless $\kappa$-statistic of the best algorithm\textgreater, \ldots  } \\ 
%
%\basecassAttackAndResults{\textless Whether an attack seems possible based on previous findings\textgreater}{\textless How the attack works\textgreater}{\textless Real success rate of attack\textgreater}{\textless Possible observations\textgreater} \\ 
%
%\basecassConclusion{\textless Possible list of weaknesses found, in decreasing order of importance \textgreater}{\textless If the CAPTCHA can be considered bypassed, and if so, the success ratio of the attack \textgreater}{\textless If any plausible work-arounds would prevent this and similar attacks \textgreater} \\ 
%	
%
%      
%    \caption{BASECASS test table.}
%    \label{tab:TestBASECASSSummary}
%  \end{longtable}






BASECASS is a framework to check a basic level of security on new CAPTCHA proposals. It is explained in the PhD thesis "WHERE DO CAPTCHAS FAIL: A STUDY IN COMMON PITFALLS IN CAPTCHA DESIGN AND HOW TO AVOID THEM", by Carlos J. Hern\'{a}ndez-Castro.

Here we present an empty BASECASS table for reference. This table can be used as a template by any cybersecurity practitioner when applying BASECASS to a new CAPTCHA.

\setlength\tabcolsep{5.5pt} % default value: 6pt

\begin{longtable}{@{\extracolsep{\fill}}l@{}} %{@{}l@{}}

    \caption{BASECASS template.} \\
  
    \basecassTitle
    {} % Main title
    {} % Name of the CAPTCHA
    {} % Description of the challenge
     \\
    
    \ChallengeDomain
    {} % Base-problem type
    {} % Base-problem size
    {} % CAPTCHA (real) problem domain
    {} % CAPTCHA problem size
    {} % CAPTCHA problem distribution of challenges (if possible to study)
     \\

	\basecassAnswerDomain
	{} % Maximum theoretical range
	{} % Actual range in CAPTCHA
	{} % Ratio maximum/real
	{} % Explanation
	{} % Distribution of correct answers (if possible to study and significant result)
	 \\

	\basecassChallengeAnswerDomainConclusions
	{} % Attack is possible base in domains: yes/no
	{} % why yes/no
	{} % Attack description
	{} % Attack success rate
	 \\

	\basecassMetrics
	{} % Denoising techniques prior to metrics
	{} % TRansforms prior to metrics
	{} % Generic metric #1
	{\vspace*{42px},} % Other generic metrics, comma separated
	{} % Order metric #1
	{\vspace*{42px},} % Other order metrics, comma separated
	{} % Specific metric #1
	{\vspace*{42px},} % Other specific metrics, comma separated
	 \\


	\basecassTestOfMetrics
		{} % Is possible an attack using one or a combination of the metrics?
		{} % Attack description
		{} % Attack success rate
		{}
		{}
		 \\
	
\basecassDataPreparation
	{\vspace*{22px}} % Training set size
	{\vspace*{22px}} % Training set balance among classes
	{\vspace*{22px}} % Training set comments
	\\

\basecassStatisticalAnalysis
	{} % Correlations found
	{} % Regressions or orther lineal relationships found
	 \\

\basecassMLAnalysis
	{} % Selection criteria for best ML algorithm
	{} % Best algorithm(s)
	{} % Accuracy/ies
	{} % kappa-statistic/s
	 \\

\basecassSMLAttackAndResults
	{} % Is a ML attack possible?: yes/no
	{} % Attack description
	{} % Attack success rate
	{} % Observations
	 \\

\basecassMLParameterAnalysis
	{} % Combination of parameter, value, and interesting ML results
	{}
	{}
	 \\ 

\basecassAttackAndResults
	{} % Is any attack possible?: yes/no
	{} % Attack description
	{} % Attack success rate
	{} % Observations
	 \\ 

\basecassConclusion
	{} % Weaknesses found
	{} % Was the CAPTCHA broken? If so, indicate success rate
	{} % Possible improvements or work-arounds on the CAPTCHA
	 \\ 
	

    \label{tab:BASECASSTemplate}

\end{longtable}


\end{document} 











